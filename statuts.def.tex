% Définitions des couleurs de la numérotation des titres
% Titre
\definecolor{couleurTitre}{HTML}{cc0000}
% Article
\definecolor{couleurArticle}{HTML}{16537e}
% Section
\definecolor{couleurSection}{HTML}{6aa84f}

% Redéfinition de l'affichage du titre de section = titre
\titleformat{\section}{\normalfont\Huge\bfseries}{\color{couleurTitre} TITRE \thesection}{1em}{}
% Numérotation des sections en chiffres romains
\renewcommand\thesection{\Roman{section}}

% Redéfinition de l'affichage du titre de sous-section = article
\titleformat{\subsection}{\normalfont\huge\bfseries}{\color{couleurArticle} Article \arabic{subsection}}{1em}{}

% Redéfinition de l'affichage du titre de sous-sous-section = Section
\titleformat{\subsubsection}{\normalfont\large\bfseries}{\color{couleurSection} Section \thesubsubsection}{1em}{}
% Décalage du titre vers la droite + marge au dessus
\titlespacing{\subsubsection}{8pt}{5pt}{0pt}
% Numérotation des sous-sous-sections en lettre associé avec le numéro de sous-section
\renewcommand\thesubsubsection{\arabic{subsection}\Alph{subsubsection}}

% Le numéro de sous-section n'est pas remis à 0 à chaque section
\counterwithout{subsection}{section}

% Pour l'affichage des références avec cref
% Section = Titre
\crefname{section}{titre}{titres}
\Crefname{section}{Titres}{Titres}
% Sous section = Article
\crefname{subsection}{article}{articles}
\Crefname{subsection}{Article}{Articles}
% Sous-sous section = Section
\crefname{subsubsection}{section}{Section}
\Crefname{subsubsection}{Section}{Sections}